\section{Methods}
In general the data has to be sorted by increasing time to failure.

\subsection{Weibull Paper}

\subsection{Median Rank Regression}
The Median Rank method uses uncensored data. Hence it is not able to incorporate information about survivors. 
We use sorting of the data to assign ranks $i$ in an ascending order. The size of the sample is denoted by $n$. The estimator is then given by
$$ F_i = \frac{i - 0.3}{n + 0.4}.$$

\subsection{Sudden Death Method}
The Sudden Death Method is the only method in this project using multiple samples of sizes $n_i$ and $N = \sum_{i}n_i$. The idea is to do multiple experiments (of the same design) and end the experiment after the first failure. Hence the data consists of one failed unit and $n_i - 1$ survivors per sample. We sort the time to failure of the failed units. The ranking of the failed units is then given by
$$ j_i = j_{i-1} + \Delta j_{i,i-1}, $$
where
$$ \Delta j_{i, i-1} = \frac{N + 1 - j_{i-1}}{N + 1 - \sum_{k=1}^{i-1} n_k}.$$
The quantile estimator is finally given by
$$ F_i = \frac{j_i - 0.3}{n + 0.4}. $$

\subsection{Kaplan-Meier Method}
The Kaplan-Meier Method can be used in right-censored data. Again, the data is sorted by time to failure. KM can be used when there are multiple events (survivals and fails) at one time.
The estimator is given by
$$ F_i = 1 - \prod_{t_i \leq t} \left( \frac{n_i - d_i}{n_i} \right), $$
where $n_i$ is the number of units in the experiment up until $t$ and $d_i$ is the number of failed units at the time $t$.
If the last event consists of failures only, we should use the adjusted estimator
$$ F_i = 1 - \prod_{t_i \leq t} \left( \frac{n_i - d_i + 1}{n_i + 1} \right) $$
to avoid a quantile $F_i = 1$. This would lead to an error when transforming the estimation for the linear regression.

\subsection{Nelson Method}
The Nelson Method can be used on right-censored data. We first assign an inverse Ranking $r_i$ for all of the samples in our data. Then, we compute the failure rate by $\lambda_{\text{Nel}, i}(t) = \frac{1}{r_i}$ for all failed units. The cumulative failure rate is given by $H_{\text{Nel}, i}(t) = \sum_{j=1}^{i} \lambda_{\text{Nel}, i}$. Finally we use the estimator
$$ F_i(t) = 1 - e^{-H_{\text{Nel}, i}(t)}. $$
Due to an strictly monotonously increasing $\lambda_{\text{Nel}}$, this method is only suitable for Weibull distributions with $b>0$. (For a derivation of $\lambda_{\text{Nel}}$ please check other sources.)

\subsection{Johnson Method}
The Johnson Method also incorporates information about survivors. The ranking is given by the following:
$$ j_i = j_{i-1} + \Delta j_{i,i-1}, $$
$j_0 = 0$ and
$$ \Delta j_{i,i-1} = \left( \frac{N + 1 - j_{i-1}}{N + 1 - K} \right) \cdot n_F, $$
where $n_F$ is the number of failed units at time $t$ and $K$ is the sum of all units that are removed (failures and survivors) from the experiment before time $t$. The quantile estimator is then given by
$$ F_i = \frac{j_i - 0.3}{n + 0.4}. $$

\subsection{Linear Regression for Estimation of Weibull Parameters}